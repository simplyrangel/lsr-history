% A snapshot of Lawrence Sullivan Ross' history.
%
\documentclass[12pt]{article}
\usepackage[margin=1in]{geometry}
\usepackage{graphicx} %enable image embedding
\usepackage{longtable}
\usepackage{csquotes} %display quotes
\usepackage{url}

% bibliography
% basic information here: https://www.overleaf.com/learn/latex/biblatex_citation_styles
\usepackage[
    backend=bibtex,
    citestyle=numeric
]{biblatex}
\addbibresource{references}

% document misc: 
\parindent 0pt % no indents for new paragraphs
\renewcommand{\familydefault}{\sfdefault}

% table of contents:
\usepackage{color}
\usepackage{hyperref}
\hypersetup{
    colorlinks=true,
    linktoc=all,
    linkcolor=black,
}

% -----------------------------------------------------
% Document start.
% -----------------------------------------------------
\begin{document}
\Large{\textbf{Lawrence Sullivan Ross \\}}
\large{An incomplete, brief biography with emphasis on racial attitudes \\}
\rule{\textwidth}{1pt}

\section{Summary}

% table of contents:
\parskip 0.5ex
\newpage
\tableofcontents
\parskip 2.0ex

% -----------------------------------------------------
% Purpose and scope.
% -----------------------------------------------------
\newpage
\section{Document purpose \& scope}

% -----------------------------------------------------
% Frequently asked questions.
% -----------------------------------------------------
\newpage
\section{FAQ's and attempts to answer them}

\subsection{On slavery}
\textbf{Did LSR or his family ever own slaves? \\ }
The preliminary search found no evidence that Lawrence Ross ever owned, traded, or sold slaves. According to Page \cite[pg.49]{page}, the 1860 McLennan County Slave Census, and the 1862, 1863, and 1864 McLennan County Tax Rolls, do not list Lawrence Ross as a slave owner.

The preliminary search found evidence that Lawrence Ross' parents owned slaves. According to Page \cite[pg.49]{page}, the 1830 Lincoln County, MO Census lists Shapely Ross, Lawrence Ross' father, with Shapely Ross' mother and 18 slaves. Page \cite[pg.50--51]{page} lists several historical sources that state Shapely Ross first came to Texas after a physical altercation with a lawyer over a runaway slave made him flee the local authorities. 

Page \cite[pg.51--55]{page} found evidence that Shapely Ross owned several slaves between 1840 and the late 1860's. 

\textbf{What were LSR's thoughts on slavery before the Civil War? \\ }
The preliminary search found little evidence on Lawrence Ross' opinion on slavery before the Civil War. Page \cite[pg. 59]{page} reasons that whatever Lawrence Ross' true motivations were, Lawrence Ross must have known that he was supporting a government dedicated to the preservation of slavery. Vaughan \cite{vaughan:email} searched some of Lawrence Ross' correspondence archived at the Baylor University Library's Ross collection and found no explicit statements on Lawrence Ross' ideas or practices regarding slavery. 

The preliminary search suggests slavery was not Lawrence Ross' primary motivation for service in the Confederate Army. At an October 1892 speech to Confederate Army veterans recorded in the Galvaston Daily News \cite{galvastondaily}, Lawrence Ross said slavery was not a primary motivator for the "great majority of Confederates":

\begin{displayquote}
“In behalf of thousands of old Confederates I want to record the fact today, that while slavery was undoubtedly an element which served to keep the public mind of the country like an angry sea that was continually casting up mire and dirt, it did not represent the principles for which the great majority of Confederates contended.  As an evidence of this fact I simply illustrate a general truth by saying that not 100 of the 1200 men composing the regiment in which I enlisted at the commencement of the struggle ever owned or expected to own a slave.  Very many of them had not left their former northern homes long enough to entitle them to vote here and yet when their adopted state took the fatal step, though subjected to the severest ordeal through which men wore ever called upon to pass, they determined to share her fate and they adhered to her cause with consistent and unshaken fidelity until it perished by war.” 

-- Lawrence Sullivan Ross, at an October 26, 1892 Confederate Army veterans reunion held at the State Fair.
\end{displayquote}

Page \cite[pg. 59--60]{page} references Charles' Grear's PhD dissertation "Texans to the Front: Why Texans Fought in the Civil War" to to quote several letters between Lawrence Ross and his wife in which Lawrence Ross wrote his sense of duty to country kept him in the field even though he longed for home. 

\textbf{What were LSR's thoughts on slavery after the Civil War? \\ }
Several sources suggest Lawrence Ross considered slavery a dead part of the South that should never return. Lawrence Ross wrote in his post Civil War presidential pardon request \cite{pardonrequest} that:

\begin{displayquote}
He would further say that he regards the slavery question as finally settled, and would view any attempt to reestablish slavery in the South as injudicious \& impolitic.  He believes that the People of the South should regard the question as settled for ever, and that it devolves upon the Southern States in their respective conventions to so provide in their organic laws.

-- L.S. Ross application for special pardon to the President of the United States, August 4, 1865.
\end{displayquote}

Page \cite[pg. 161--167]{page} lists several speeches and documents where Lawrence Ross said the South's future lay in continued reunification with the North and adherence to the Union's federal laws and government. A speech Lawrence Ross made to the annual reunion of Hood's Brigade \cite{ftworthgazette} said:

\begin{displayquote}
It is a remarkable fact that those who bore the brunt of battle were the first to forget old animosities and consign to oblivion obsolete issues.  They saw that nothing but sorrow and shame and the loss of the respect of the world was to be gained by perpetuating the bitterness of past strife.  And impelled by a spirit of patriotism they were willing by all possible methods to create and give utterance to a public sentiment which would best conserve our common institutions and restore that fraternal concord in which the war of the revolution left us and the Federal constitution found us.

-- Lawrence Ross' Address to the Annual reunion of Hood's Brigade, as recorded in the June 28, 1887 Fort Worth Daily Gazette.
\end{displayquote}

\subsection{On the Confederacy}
\textbf{What did LSR think about the Confederate States of America? \\}

\textbf{Did LSR serve in the Confederate Army? \\ }

\subsection{Attitude on race}
\textbf{Was LSR a member of the Klu Klux Klan? \\ }

\textbf{What was LSR's attitude towards racial minorities, particularly African Americans? \\ }

\textbf{What did the contemporary black community think of LSR? \\ }

\subsection{On Texas A\&M and Prairie View A\&M}
\textbf{When was LSR's statue installed on Texas A\&M's campus? \\ }

\textbf{Was LSR involved with Prairie View A\&M? \\ }

% -----------------------------------------------------
% Timeline.
% -----------------------------------------------------
\newpage
\section{Timeline of notable events}

% -----------------------------------------------------
% Family history.
% -----------------------------------------------------
\newpage
\section{Family history}

% -----------------------------------------------------
% Civil war.
% -----------------------------------------------------
\newpage
\section{The Civil War}

\subsection{Personal views on the Confederacy and slavery}

\subsection{An incomplete service record}

\subsection{Notable military engagements}

% -----------------------------------------------------
% Public office.
% -----------------------------------------------------
\newpage
\section{Public office}

\subsection{Local politics}

\subsection{Governor of Texas}

% -----------------------------------------------------
% Texas A&M.
% -----------------------------------------------------
\newpage
\section{Texas A\&M University President}

% -----------------------------------------------------
% Death.
% -----------------------------------------------------
\newpage
\section{Death and legacy}

% -----------------------------------------------------
% References.
% -----------------------------------------------------
\newpage
\printbibliography

% -----------------------------------------------------
% End document.
% -----------------------------------------------------
\end{document}







