% This file was autogenerated by timeline.py.
%
\begin{longtable}{|p{0.06\linewidth}|p{0.4\linewidth}|p{0.4\linewidth}|p{0.1\linewidth}|} 
\caption{High level timeline of Lawrence Ross' family and personal life.} \\ 
\hline 
\textbf{Year} & \textbf{State or national history} & \textbf{LSR personal or family history} & \textbf{Source} \\ 
\hline 
1835 & Texas revolution begins 
(October 2, 1835) &   &   \\ 
\hline 
1836 & The Battle of the Alamo 
(February 23 through March 6, 1836) &   &   \\ 
\hline 
1836 & Texas Declaration of Independence is adopted (March 2, 1836) &   &   \\ 
\hline 
1836 & Texas is annexed into the United States as the 28th state in the Union (December 29, 1845) &   &   \\ 
\hline 
1838 &   & LSR is born to Shapely Ross and Catherine Fulkerson in Iowa (September 27, 1838) & \cite{rosspapersummary} \\ 
\hline 
1838 &   & Shapely Ross runs to Texas after physical altercation with a lawyer over a runaway slave (sometime in 1838). Catherine Fulkerson follows with rest of family shortly after.  & \cite[pg. 53]{page} \\ 
\hline 
1849 &   & Ross family moves to Waco permanently (1849) & \cite{rosspapersummary} \\ 
\hline 
1859 &   & LSR graduates from Wesleyan University in Alabama (1859) & \cite{rosspapersummary} \\ 
\hline 
1861 &   & LSR resigns from active duty Texas Ranger service (1861; immediately before U.S. Civil War begins, but no date given) & \cite{rosspapersummary} \\ 
\hline 
1861 & United States Civil War begins (April 12, 1861) &   &   \\ 
\hline 
1861 &   & LSR marries Elizabeth Dorothy Tinsley (May 28, 1861) & \cite{rosspapersummary} \\ 
\hline 
1861 &   & Private LSR elected Major in the 6th Regiment of Texas Cavalry by his peers & \cite[pg. 36 and 37]{texasbrigade} \\ 
\hline 
1862 & Battle of Antietam (September 17, 1862). Bloodiest day in US military history with 22,717 dead, wounded, or missing. &   &   \\ 
\hline 
1863 & Emancipation Proclamation issued by President Abraham Lincoln (January 1, 1863) &   &   \\ 
\hline 
1863 & Battle of Gettysburg (July 1 thru 3, 1863) &   &   \\ 
\hline 
1863 &   & LSR promoted to Brigadier General (December 1863). Ross was 25 years of age. Ross commanded a cavalry brigade in the Army of Tennessee. & \cite{rosspapersummary} \\ 
\hline 
1865 & United States Civil War ends (April 9, 1865) &   &   \\ 
\hline 
1865 & President Abraham Lincoln is shot by John Wilkes Booth (April 14, 1865) and dies shortly after (April 15, 1865) &   &   \\ 
\hline 
1865 & Union general Gordon Granger reads ‘General Order No. 3’ to the public in Galvaston, TX, announcing total emancipation of those held as slaves (June 18, 1865; commonly known as Juneteenth) &   &   \\ 
\hline 
1865 & 13th Amendment to the US Constitution is adopted, formally abolishing slavery (December 18, 1865) &   &   \\ 
\hline 
1873 &   & LSR elected sheriff of McLennan County, Texas (1873) & \cite{rosspapersummary} \\ 
\hline 
1876 &   & LSR attends the Texas Constitutional Convention as a delegate from Central Texas (1876) & \cite{rosspapersummary} \\ 
\hline 
1880 &   & LSR elected Texas state senator, going on to serve one term (1880) & \cite{rosspapersummary} \\ 
\hline 
1887 &   & LSR becomes Governor of Texas (January 18, 1887). Ross would go on to serve two terms as governor, leaving office January 20, 1891.  & \cite{tsl} \\ 
\hline 
1889 &   & Shapely Ross, father to LSR, dies (September 17, 1889) & \cite{rosspapersummary} \\ 
\hline 
1891 &   & LSR leaves the office of the Governor of the State of Texas (January 20, 1891 & \cite{tsl} \\ 
\hline 
1891 &   & LSR accepts position as President of the Agricultural and Mechanical College of Texas (1891) & \cite{rosspapersummary} \\ 
\hline 
1898 &   & LSR dies at home (January 3, 1898). Ross was still college president at the time.  & \cite{rosspapersummary} \\ 
\hline 
\end{longtable}